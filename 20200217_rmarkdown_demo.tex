\documentclass[]{article}
\usepackage{lmodern}
\usepackage{amssymb,amsmath}
\usepackage{ifxetex,ifluatex}
\usepackage{fixltx2e} % provides \textsubscript
\ifnum 0\ifxetex 1\fi\ifluatex 1\fi=0 % if pdftex
  \usepackage[T1]{fontenc}
  \usepackage[utf8]{inputenc}
\else % if luatex or xelatex
  \ifxetex
    \usepackage{mathspec}
  \else
    \usepackage{fontspec}
  \fi
  \defaultfontfeatures{Ligatures=TeX,Scale=MatchLowercase}
\fi
% use upquote if available, for straight quotes in verbatim environments
\IfFileExists{upquote.sty}{\usepackage{upquote}}{}
% use microtype if available
\IfFileExists{microtype.sty}{%
\usepackage{microtype}
\UseMicrotypeSet[protrusion]{basicmath} % disable protrusion for tt fonts
}{}
\usepackage[margin=1in]{geometry}
\usepackage{hyperref}
\hypersetup{unicode=true,
            pdftitle={R-Ladies Helsinki February Event},
            pdfauthor={Hazel KAVILI},
            pdfborder={0 0 0},
            breaklinks=true}
\urlstyle{same}  % don't use monospace font for urls
\usepackage{color}
\usepackage{fancyvrb}
\newcommand{\VerbBar}{|}
\newcommand{\VERB}{\Verb[commandchars=\\\{\}]}
\DefineVerbatimEnvironment{Highlighting}{Verbatim}{commandchars=\\\{\}}
% Add ',fontsize=\small' for more characters per line
\usepackage{framed}
\definecolor{shadecolor}{RGB}{248,248,248}
\newenvironment{Shaded}{\begin{snugshade}}{\end{snugshade}}
\newcommand{\AlertTok}[1]{\textcolor[rgb]{0.94,0.16,0.16}{#1}}
\newcommand{\AnnotationTok}[1]{\textcolor[rgb]{0.56,0.35,0.01}{\textbf{\textit{#1}}}}
\newcommand{\AttributeTok}[1]{\textcolor[rgb]{0.77,0.63,0.00}{#1}}
\newcommand{\BaseNTok}[1]{\textcolor[rgb]{0.00,0.00,0.81}{#1}}
\newcommand{\BuiltInTok}[1]{#1}
\newcommand{\CharTok}[1]{\textcolor[rgb]{0.31,0.60,0.02}{#1}}
\newcommand{\CommentTok}[1]{\textcolor[rgb]{0.56,0.35,0.01}{\textit{#1}}}
\newcommand{\CommentVarTok}[1]{\textcolor[rgb]{0.56,0.35,0.01}{\textbf{\textit{#1}}}}
\newcommand{\ConstantTok}[1]{\textcolor[rgb]{0.00,0.00,0.00}{#1}}
\newcommand{\ControlFlowTok}[1]{\textcolor[rgb]{0.13,0.29,0.53}{\textbf{#1}}}
\newcommand{\DataTypeTok}[1]{\textcolor[rgb]{0.13,0.29,0.53}{#1}}
\newcommand{\DecValTok}[1]{\textcolor[rgb]{0.00,0.00,0.81}{#1}}
\newcommand{\DocumentationTok}[1]{\textcolor[rgb]{0.56,0.35,0.01}{\textbf{\textit{#1}}}}
\newcommand{\ErrorTok}[1]{\textcolor[rgb]{0.64,0.00,0.00}{\textbf{#1}}}
\newcommand{\ExtensionTok}[1]{#1}
\newcommand{\FloatTok}[1]{\textcolor[rgb]{0.00,0.00,0.81}{#1}}
\newcommand{\FunctionTok}[1]{\textcolor[rgb]{0.00,0.00,0.00}{#1}}
\newcommand{\ImportTok}[1]{#1}
\newcommand{\InformationTok}[1]{\textcolor[rgb]{0.56,0.35,0.01}{\textbf{\textit{#1}}}}
\newcommand{\KeywordTok}[1]{\textcolor[rgb]{0.13,0.29,0.53}{\textbf{#1}}}
\newcommand{\NormalTok}[1]{#1}
\newcommand{\OperatorTok}[1]{\textcolor[rgb]{0.81,0.36,0.00}{\textbf{#1}}}
\newcommand{\OtherTok}[1]{\textcolor[rgb]{0.56,0.35,0.01}{#1}}
\newcommand{\PreprocessorTok}[1]{\textcolor[rgb]{0.56,0.35,0.01}{\textit{#1}}}
\newcommand{\RegionMarkerTok}[1]{#1}
\newcommand{\SpecialCharTok}[1]{\textcolor[rgb]{0.00,0.00,0.00}{#1}}
\newcommand{\SpecialStringTok}[1]{\textcolor[rgb]{0.31,0.60,0.02}{#1}}
\newcommand{\StringTok}[1]{\textcolor[rgb]{0.31,0.60,0.02}{#1}}
\newcommand{\VariableTok}[1]{\textcolor[rgb]{0.00,0.00,0.00}{#1}}
\newcommand{\VerbatimStringTok}[1]{\textcolor[rgb]{0.31,0.60,0.02}{#1}}
\newcommand{\WarningTok}[1]{\textcolor[rgb]{0.56,0.35,0.01}{\textbf{\textit{#1}}}}
\usepackage{longtable,booktabs}
\usepackage{graphicx,grffile}
\makeatletter
\def\maxwidth{\ifdim\Gin@nat@width>\linewidth\linewidth\else\Gin@nat@width\fi}
\def\maxheight{\ifdim\Gin@nat@height>\textheight\textheight\else\Gin@nat@height\fi}
\makeatother
% Scale images if necessary, so that they will not overflow the page
% margins by default, and it is still possible to overwrite the defaults
% using explicit options in \includegraphics[width, height, ...]{}
\setkeys{Gin}{width=\maxwidth,height=\maxheight,keepaspectratio}
\IfFileExists{parskip.sty}{%
\usepackage{parskip}
}{% else
\setlength{\parindent}{0pt}
\setlength{\parskip}{6pt plus 2pt minus 1pt}
}
\setlength{\emergencystretch}{3em}  % prevent overfull lines
\providecommand{\tightlist}{%
  \setlength{\itemsep}{0pt}\setlength{\parskip}{0pt}}
\setcounter{secnumdepth}{0}
% Redefines (sub)paragraphs to behave more like sections
\ifx\paragraph\undefined\else
\let\oldparagraph\paragraph
\renewcommand{\paragraph}[1]{\oldparagraph{#1}\mbox{}}
\fi
\ifx\subparagraph\undefined\else
\let\oldsubparagraph\subparagraph
\renewcommand{\subparagraph}[1]{\oldsubparagraph{#1}\mbox{}}
\fi

%%% Use protect on footnotes to avoid problems with footnotes in titles
\let\rmarkdownfootnote\footnote%
\def\footnote{\protect\rmarkdownfootnote}

%%% Change title format to be more compact
\usepackage{titling}

% Create subtitle command for use in maketitle
\providecommand{\subtitle}[1]{
  \posttitle{
    \begin{center}\large#1\end{center}
    }
}

\setlength{\droptitle}{-2em}

  \title{R-Ladies Helsinki February Event}
    \pretitle{\vspace{\droptitle}\centering\huge}
  \posttitle{\par}
    \author{Hazel KAVILI}
    \preauthor{\centering\large\emph}
  \postauthor{\par}
      \predate{\centering\large\emph}
  \postdate{\par}
    \date{1/27/2020}


\begin{document}
\maketitle

\hypertarget{spotify-songs}{%
\subsection{Spotify Songs}\label{spotify-songs}}

We will work on a
\emph{\href{https://github.com/rfordatascience/tidytuesday/blob/master/data/2020/2020-01-21/readme.md}{TidyTuesday}}
dataset today. \emph{(Try to check out the TidyTuesday concept after the
event! You'll love it!)}

\hypertarget{load-libraries}{%
\paragraph{Load libraries}\label{load-libraries}}

\begin{Shaded}
\begin{Highlighting}[]
\KeywordTok{library}\NormalTok{(tidyverse)}
\KeywordTok{library}\NormalTok{(lubridate)}
\KeywordTok{library}\NormalTok{(knitr)}
\end{Highlighting}
\end{Shaded}

\hypertarget{read-the-data-set-from-source}{%
\paragraph{Read the data set from
source}\label{read-the-data-set-from-source}}

\begin{Shaded}
\begin{Highlighting}[]
\NormalTok{spotify_songs <-}\StringTok{ }
\StringTok{  }\NormalTok{readr}\OperatorTok{::}\KeywordTok{read_csv}\NormalTok{(}\StringTok{'https://raw.githubusercontent.com/rfordatascience/tidytuesday/master/data/2020/2020-01-21/spotify_songs.csv'}\NormalTok{)}
\end{Highlighting}
\end{Shaded}

\hypertarget{start-exploring}{%
\paragraph{Start Exploring}\label{start-exploring}}

\emph{glimpse} function makes it possible to see every column and some
observations in a data frame.

\begin{Shaded}
\begin{Highlighting}[]
\KeywordTok{glimpse}\NormalTok{(spotify_songs)}
\end{Highlighting}
\end{Shaded}

\begin{verbatim}
## Observations: 32,833
## Variables: 23
## $ track_id                 <chr> "6f807x0ima9a1j3VPbc7VN", "0r7CVbZTWZ...
## $ track_name               <chr> "I Don't Care (with Justin Bieber) - ...
## $ track_artist             <chr> "Ed Sheeran", "Maroon 5", "Zara Larss...
## $ track_popularity         <dbl> 66, 67, 70, 60, 69, 67, 62, 69, 68, 6...
## $ track_album_id           <chr> "2oCs0DGTsRO98Gh5ZSl2Cx", "63rPSO264u...
## $ track_album_name         <chr> "I Don't Care (with Justin Bieber) [L...
## $ track_album_release_date <chr> "2019-06-14", "2019-12-13", "2019-07-...
## $ playlist_name            <chr> "Pop Remix", "Pop Remix", "Pop Remix"...
## $ playlist_id              <chr> "37i9dQZF1DXcZDD7cfEKhW", "37i9dQZF1D...
## $ playlist_genre           <chr> "pop", "pop", "pop", "pop", "pop", "p...
## $ playlist_subgenre        <chr> "dance pop", "dance pop", "dance pop"...
## $ danceability             <dbl> 0.748, 0.726, 0.675, 0.718, 0.650, 0....
## $ energy                   <dbl> 0.916, 0.815, 0.931, 0.930, 0.833, 0....
## $ key                      <dbl> 6, 11, 1, 7, 1, 8, 5, 4, 8, 2, 6, 8, ...
## $ loudness                 <dbl> -2.634, -4.969, -3.432, -3.778, -4.67...
## $ mode                     <dbl> 1, 1, 0, 1, 1, 1, 0, 0, 1, 1, 1, 1, 1...
## $ speechiness              <dbl> 0.0583, 0.0373, 0.0742, 0.1020, 0.035...
## $ acousticness             <dbl> 0.10200, 0.07240, 0.07940, 0.02870, 0...
## $ instrumentalness         <dbl> 0.00e+00, 4.21e-03, 2.33e-05, 9.43e-0...
## $ liveness                 <dbl> 0.0653, 0.3570, 0.1100, 0.2040, 0.083...
## $ valence                  <dbl> 0.518, 0.693, 0.613, 0.277, 0.725, 0....
## $ tempo                    <dbl> 122.036, 99.972, 124.008, 121.956, 12...
## $ duration_ms              <dbl> 194754, 162600, 176616, 169093, 18905...
\end{verbatim}

\hypertarget{exploring-data}{%
\paragraph{Exploring data}\label{exploring-data}}

Some songs are duplicated, because they're in different albums or in
different playlist. I wonder, how many distinct tracks there are for
each artist, and I'll look for top 20:

\begin{Shaded}
\begin{Highlighting}[]
\NormalTok{artists_tracks <-}\StringTok{ }\NormalTok{spotify_songs }\OperatorTok\StringTok{ }
\StringTok{  }\KeywordTok{distinct}\NormalTok{(track_id, }\DataTypeTok{.keep_all =} \OtherTok{TRUE}\NormalTok{) }\OperatorTok\StringTok{ }
\StringTok{  }\KeywordTok{count}\NormalTok{(track_artist, }\DataTypeTok{sort =} \OtherTok{TRUE}\NormalTok{) }\OperatorTok
\StringTok{  }\KeywordTok{top_n}\NormalTok{(}\DataTypeTok{n =} \DecValTok{20}\NormalTok{, }\DataTypeTok{wt =}\NormalTok{ n)}
\end{Highlighting}
\end{Shaded}

I want to see results in a stylish table:

\begin{Shaded}
\begin{Highlighting}[]
\KeywordTok{head}\NormalTok{(artists_tracks) }\OperatorTok\StringTok{ }
\StringTok{  }\KeywordTok{kable}\NormalTok{(}\DataTypeTok{align =} \StringTok{"lccrr"}\NormalTok{,  }\DataTypeTok{caption =} \StringTok{"Top 10 artists with most tracks"}\NormalTok{)}
\end{Highlighting}
\end{Shaded}

\begin{longtable}[]{@{}lc@{}}
\caption{Top 10 artists with most tracks}\tabularnewline
\toprule
track\_artist & n\tabularnewline
\midrule
\endfirsthead
\toprule
track\_artist & n\tabularnewline
\midrule
\endhead
Queen & 130\tabularnewline
Martin Garrix & 87\tabularnewline
Don Omar & 84\tabularnewline
David Guetta & 81\tabularnewline
Dimitri Vegas \& Like Mike & 68\tabularnewline
Drake & 68\tabularnewline
\bottomrule
\end{longtable}

Let's create a plot by using this data:

\begin{Shaded}
\begin{Highlighting}[]
\NormalTok{artist_plot <-}\StringTok{ }\KeywordTok{ggplot}\NormalTok{(}\DataTypeTok{data =}\NormalTok{ artists_tracks, }\KeywordTok{aes}\NormalTok{(}\DataTypeTok{x =} \KeywordTok{reorder}\NormalTok{(track_artist, n), }\DataTypeTok{y =}\NormalTok{ n)) }\OperatorTok{+}
\StringTok{  }\KeywordTok{geom_bar}\NormalTok{(}\DataTypeTok{stat =} \StringTok{'identity'}\NormalTok{) }\OperatorTok{+}
\StringTok{  }\KeywordTok{coord_flip}\NormalTok{() }\OperatorTok{+}
\StringTok{  }\KeywordTok{labs}\NormalTok{(}\DataTypeTok{title =} \StringTok{'Top 20 artists with most tracks in list'}\NormalTok{, }
       \DataTypeTok{x =} \StringTok{'Artists'}\NormalTok{, }
       \DataTypeTok{y =} \StringTok{'Number of tracks'}\NormalTok{)}

\NormalTok{artist_plot}
\end{Highlighting}
\end{Shaded}

\includegraphics{20200217_rmarkdown_demo_files/figure-latex/unnamed-chunk-6-1.pdf}

I realise some artists released so many albums during years and I wonder
is the longest time passed since they released their last album.

\begin{Shaded}
\begin{Highlighting}[]
\NormalTok{album_release_years <-}\StringTok{ }
\StringTok{  }\NormalTok{spotify_songs }\OperatorTok\StringTok{ }
\StringTok{  }\KeywordTok{mutate}\NormalTok{(}\DataTypeTok{release_year =} \KeywordTok{as.numeric}\NormalTok{(}\KeywordTok{str_sub}\NormalTok{(track_album_release_date, }\DecValTok{1}\NormalTok{, }\DecValTok{4}\NormalTok{))) }\OperatorTok\StringTok{ }\CommentTok{#get only year information}
\StringTok{  }\KeywordTok{distinct}\NormalTok{(track_id, }\DataTypeTok{.keep_all =} \OtherTok{TRUE}\NormalTok{) }\OperatorTok\StringTok{ }
\StringTok{  }\KeywordTok{distinct}\NormalTok{(track_name, track_artist, }\DataTypeTok{.keep_all =} \OtherTok{TRUE}\NormalTok{) }\OperatorTok\StringTok{ }
\StringTok{  }\KeywordTok{group_by}\NormalTok{(track_artist) }\OperatorTok\StringTok{ }
\StringTok{  }\KeywordTok{mutate}\NormalTok{(}\DataTypeTok{first_release_year =} \KeywordTok{min}\NormalTok{(release_year),}
         \DataTypeTok{last_release_year =} \KeywordTok{max}\NormalTok{(release_year),}
         \DataTypeTok{year_diff =}\NormalTok{ last_release_year }\OperatorTok{-}\StringTok{ }\NormalTok{first_release_year) }\OperatorTok
\StringTok{  }\KeywordTok{ungroup}\NormalTok{()  }\OperatorTok\StringTok{ }
\StringTok{  }\KeywordTok{mutate}\NormalTok{(}\DataTypeTok{track_artist =} \KeywordTok{fct_reorder}\NormalTok{(track_artist, year_diff)) }
\end{Highlighting}
\end{Shaded}

\begin{Shaded}
\begin{Highlighting}[]
\NormalTok{album_release_years }\OperatorTok\StringTok{  }
\StringTok{  }\KeywordTok{filter}\NormalTok{(year_diff }\OperatorTok{>}\StringTok{ }\DecValTok{50}\NormalTok{) }\OperatorTok\StringTok{ }
\StringTok{  }\KeywordTok{ggplot}\NormalTok{() }\OperatorTok{+}
\StringTok{  }\KeywordTok{geom_path}\NormalTok{(}\KeywordTok{aes}\NormalTok{(}\DataTypeTok{x =}\NormalTok{ release_year, }\DataTypeTok{y =}\NormalTok{ track_artist)) }\OperatorTok{+}
\StringTok{  }\KeywordTok{geom_point}\NormalTok{(}\KeywordTok{aes}\NormalTok{(release_year, track_artist, }\DataTypeTok{color =}\NormalTok{ track_artist, }\DataTypeTok{alpha =} \FloatTok{0.1}\NormalTok{), }\DataTypeTok{size =} \DecValTok{2}\NormalTok{) }\OperatorTok{+}
\StringTok{  }\KeywordTok{labs}\NormalTok{(}\DataTypeTok{title =} \StringTok{''}\NormalTok{, }\DataTypeTok{x =} \StringTok{'Album release years'}\NormalTok{, }\DataTypeTok{y =} \StringTok{'Artists'}\NormalTok{) }\OperatorTok{+}
\StringTok{  }\KeywordTok{theme_light}\NormalTok{()}
\end{Highlighting}
\end{Shaded}

\begin{figure}
\centering
\includegraphics{20200217_rmarkdown_demo_files/figure-latex/unnamed-chunk-8-1.pdf}
\caption{Years passed since first album release}
\end{figure}

\begin{figure}
\centering
\includegraphics{https://media.giphy.com/media/xUOxf7XfmpxuSode1O/giphy.gif}
\caption{mygif}
\end{figure}


\end{document}
